%--------------------------------------------------------------------------
% @file polynomials.tex
%
% @date 09/20/16 16:22:57
% @author Martin Noblia
% @email martin.noblia@openmailbox.org
%
% @brief
% Breve resumen de polinomios
% @detail
%
% Licence:
% This program is free software: you can redistribute it and/or modify
% it under the terms of the GNU General Public License as published by
% the Free Software Foundation, either version 3 of the License, or (at
% your option) any later version.
% 
% This program is distributed in the hope that it will be useful, but
% WITHOUT ANY WARRANTY; without even the implied warranty of
% MERCHANTABILITY or FITNESS FOR A PARTICULAR PURPOSE.  See the GNU
% General Public License for more details.
% 
% You should have received a copy of the GNU General Public License
%
%---------------------------------------------------------------------------
\section{Polinomios}
\begin{defi}{Polinomios:}
   Un polinomio en la variable $x$ es de la forma $a_0 + a_1 x + a_2 x^2 + a_3 x^3 + ... + a_n x^n$
\end{defi}
\begin{defi}
   Dos polinomios son iguales si los coeficientes de igual grado son iguales: Sean $p(x) = a_0 + a_1 x + a_2 x^2 + a_3 x^3 + ... + a_n x^n$ y 
   $q(x) = b_0 + b_1 x + b_2 x^2 + b_3 x^3 + ... + b_n x^n$ entonces decimos que $p(x) = q(x)$ si y solo si:
   \begin{align}
      a_0 &= b_0\\
      a_1 &= b_1\\
      a_2 &= b_2\\
      a_3 &= b_3\\
      .\\
      .\\
      .\\
      a_n &= b_n
   \end{align}
\end{defi}

\begin{teo}{Teorema del resto:}
   Al dividir un polinomio $p(x)$ por $x-a$ se obtiene como resto $p(a)$
\end{teo}
\begin{teo}
   Un polinomio de grado $n$ no puede tener mas de $n$ raices distintas
\end{teo}

\begin{teo}{Principio de identidad de Polinomios:}
   Si dos polinomios de grado $\leq n$ valen lo mismo en $n+1$ puntos distintos entonces son iguales
\end{teo}
 \begin{teo}{Criterio de Gauss:}
    Sea $\frac{p}{q}$ una raiz racional del polinomio con coeficientes enteros: $z(x) = a_0 + a_1 x + a_2 x^2 + a_3 x^3 + ... + a_n x^n$
    Si $a_0 a_n \neq 0$ y $p$, $q$ son coprimos entonces $p|a_0$ y $q|a_n$
 \end{teo}
