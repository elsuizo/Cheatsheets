%--------------------------------------------------------------------------
% @file linear_systems.tex
%
% @date 09/26/16 09:33:12
% @author Martin Noblia
% @email martin.noblia@openmailbox.org
%
% @brief
% Resumen de sistemas lineales.
% @detail
%
% Licence:
% This program is free software: you can redistribute it and/or modify
% it under the terms of the GNU General Public License as published by
% the Free Software Foundation, either version 3 of the License, or (at
% your option) any later version.
% 
% This program is distributed in the hope that it will be useful, but
% WITHOUT ANY WARRANTY; without even the implied warranty of
% MERCHANTABILITY or FITNESS FOR A PARTICULAR PURPOSE.  See the GNU
% General Public License for more details.
% 
% You should have received a copy of the GNU General Public License
%
%---------------------------------------------------------------------------
% begin
%---------------------------------------------------------------------------
% Resumen de sistemas lineales
%---------------------------------------------------------------------------
\section{Sistemas Lineales}

\subsection{Matrices}
Una matrix $A$ de tamanio $m \times n$ es un arreglo rectangular de $m \, n$ numeros
dispuestos en $m$ filas y $n$ columnas. Por ejemplo:
$
\begin{bmatrix}
   1 & 0 & 1\\
   0 & 1 & 0 \\
   1 & 3 & 7
 \end{bmatrix}
$
Una matriz generica de tamanio $m \times n$ la anotamos: $A=(a_{ij})$ de tal manera que 
su elemento en la fila $i$ columna $j$ es $a_{ij}$. La fila $i$ de $A$ es:
$
\begin{bmatrix}
   a_{i1} & a_{i2} & a_{i3} & \cdots & a_{i\,n}
 \end{bmatrix}
$
y la columna $j$ de $A$ es:
$
\begin{bmatrix}
   a_{1j} \\ a_{2j} \\ a_{3j} \\ \vdots \\a_{n \, t}
 \end{bmatrix}
$




